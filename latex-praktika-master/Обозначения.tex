\section*{ОБОЗНАЧЕНИЯ И СОКРАЩЕНИЯ}

БД -- база данных.

ИС -- информационная система.

ИТ -- информационные технологии. 

КТС -- комплекс технических средств.

ОМТС -- отдел материально-технического снабжения. 

ПО -- программное обеспечение.

РП -- рабочий проект.

СУБД -- система управления базами данных.

ТЗ -- техническое задание.

ТП -- технический проект.

UML (Unified Modelling Language) -- язык графического описания для объектного моделирования в области разработки программного обеспечения.

HTML (HyperText Markup Language) -- язык гипертекстовой разметки.

JavaScript -- язык программирования для веб-разработки.

CSS (Cascading Style Sheets) -- каскадные таблицы стилей.

SCSS (Sassy CSS) -- препроцессор, расширяющий возможности CSS.

UI (User Interface) -- пользовательский интерфейс.

UX (User Experience) -- пользовательский опыт.

REST API (Representational State Transfer Application Programming Interface) -- интерфейс программирования приложений, использующий архитектурный стиль REST.

CRUD (Create, Read, Update, Delete) -- основные операции для работы с данными.

MVC (Model-View-Controller) -- архитектурный шаблон проектирования.

PostgreSQL -- объектно-реляционная система управления базами данных с открытым исходным кодом.

JSON (JavaScript Object Notation) -- текстовый формат обмена данными.