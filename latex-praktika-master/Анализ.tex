\section{Анализ предметной области}
\subsection{Характеристика предприятия и его деятельности}

Компания \textbf{ВТИ-Сервис} — ведущая ИТ-компания Курской области, аккредитованная Министерством цифрового развития РФ. Она занимается поставкой программного обеспечения и оборудования для автоматизации торговли, сферы услуг и промышленности, а также разработкой собственных решений. Основной целью компании является комплексная автоматизация бизнес-процессов на предприятиях различного масштаба.

Основные направления деятельности включают:

\begin{enumerate}
  \item Компания осуществляет разработку и внедрение программного обеспечения. Решения адаптированы под различные отрасли и включают:
  \begin{itemize}
    \item автоматизацию торговли;
    \item автоматизацию общественного питания;
    \item управление складским учетом.
  \end{itemize}

  \item Осуществляется поставка торгового оборудования, обеспечивающего полный цикл обслуживания клиента. В него входят:
  \begin{itemize}
    \item онлайн-кассы;
    \item сканеры штрих-кодов;
    \item принтеры чеков и этикеток;
    \item весовое оборудование.
  \end{itemize}

  \item Компания предоставляет ИТ-услуги, направленные на поддержку и развитие инфраструктуры клиентов. Среди них:
  \begin{itemize}
    \item ИТ-аутсорсинг;
    \item внедрение и сопровождение продуктов 1С;
    \item техническое обслуживание и настройка оборудования.
  \end{itemize}

  \item В области безопасности внедряются современные решения. Компания проектирует и устанавливает:
  \begin{itemize}
    \item системы видеонаблюдения;
    \item системы контроля и управления доступом;
    \item средства удалённого мониторинга.
  \end{itemize}

  \item ВТИ-Сервис активно разрабатывает собственное программное обеспечение, включая мобильные приложения. Например, приложение \textit{DM.ТОИР} используется для технического обслуживания и ремонта основных фондов предприятий.

  \item Все решения компании соответствуют действующему законодательству. Организация обеспечивает:
  \begin{itemize}
    \item передачу фискальных данных в налоговые органы;
    \item соответствие требованиям обязательной маркировки товаров.
  \end{itemize}
\end{enumerate}
