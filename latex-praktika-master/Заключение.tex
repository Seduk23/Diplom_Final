\section*{ЗАКЛЮЧЕНИЕ}
\addcontentsline{toc}{section}{ЗАКЛЮЧЕНИЕ}

Преимущества использования веб-приложений для образовательных целей заключаются в их доступности, гибкости и возможности адаптации под различные категории пользователей, включая иностранных студентов. Существенным ограничением является необходимость обеспечения мультиязычности и интуитивно понятного интерфейса, что требует дополнительных ресурсов на разработку и тестирование.

Современные образовательные учреждения активно используют информационные технологии для повышения эффективности обучения, создавая специализированные платформы, которые поддерживают самостоятельную работу студентов, предоставляют интерактивные материалы и отслеживают прогресс. Для реализации этих задач было разработано веб-приложение на основе фреймворка Django с использованием базы данных SQLite, библиотек CKEditor для редактирования контента, AOS для анимаций и Bootstrap для адаптивного интерфейса.

Основные результаты работы:

\begin{enumerate}
	\item Проведён анализ предметной области. Выявлена необходимость разработки мультиязычной образовательной платформы с поддержкой интерактивных уроков и тестов.
	\item Разработана концептуальная модель веб-приложения. Создана модель данных системы, включающая классы User, Course, Lesson, Test, Question, Answer, TestResult, Enrollment, StudentProgress, Achievement, UserAchievement. Определены требования к функциональности и интерфейсу.
	\item Осуществлено проектирование веб-приложения. Разработана архитектура серверной части на основе Django MVC. Спроектирован пользовательский интерфейс с использованием Bootstrap, CKEditor и AOS, обеспечивающий адаптивность и мультиязычность.
	\item Реализовано и протестировано веб-приложение. Проведено модульное тестирование с использованием django.test и системное тестирование, включающее проверку регистрации, создания курсов, прохождения тестов и локализации.
\end{enumerate}

Все требования, заявленные в техническом задании, были полностью реализованы, все задачи, поставленные в начале разработки проекта, успешно решены.

Готовый рабочий проект представляет собой адаптивное веб-приложение, обеспечивающее поддержку самостоятельной работы иностранных студентов при изучении JavaScript. Приложение доступно для использования в образовательных целях и может быть развёрнуто на сервере для публичного доступа.