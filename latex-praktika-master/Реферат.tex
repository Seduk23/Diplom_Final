\abstract{РЕФЕРАТ}

Объем работы равен \formbytotal{lastpage}{страниц}{е}{ам}{ам}. Работа содержит \formbytotal{figurecnt}{иллюстраци}{ю}{и}{й}, \formbytotal{tablecnt}{таблиц}{у}{ы}{}, \arabic{bibcount} библиографических источников и \formbytotal{числоПлакатов}{лист}{}{а}{ов} графического материала. Количество приложений – 2. Графический материал представлен в приложении А. Фрагменты исходного кода представлены в приложении Б.

Перечень ключевых слов: обучающая платформа, веб-приложение, Django, мультиязычность, CKEditor, AOS, JavaScript, курсы, уроки, тесты, достижения, база данных, ORM, иностранные студенты, веб-интерфейс.

Объектом разработки является веб-приложение -- обучающая платформа, предназначенная для компьютерной поддержки самостоятельной работы иностранных студентов при изучении языка программирования JavaScript, с мультиязычным интерфейсом и интерактивными инструментами для обучения.

Целью выпускной квалификационной работы является создание удобной и функциональной платформы для самостоятельного изучения языка программирования JavaScript иностранными студентами, обеспечивающей доступ к образовательным материалам, тестированию и отслеживанию прогресса, способствующий повышению вовлечённости и эффективности обучения.

В процессе создания платформы были выделены основные сущности, использованы классы и методы Django, разработаны разделы для управления курсами, аналитикой прогресса и достижений, реализованы интерактивные элементы с использованием CKEditor для редактирования учебного контента.

При разработке платформы использовался фреймворк Django с интегрированными библиотеками для аутентификации, шаблонизации, обработки форм и мультиязычности, а также CKEditor и AOS для обеспечения интерактивности и удобства работы с учебным контентом.

Разработанная платформа была успешно протестирована и внедрена в систему обучения.

\selectlanguage{english}
\abstract{ABSTRACT}
  
The volume of work is \formbytotal{lastpage}{page}{}{s}{s}. The work contains \formbytotal{figurecnt}{illustration}{}{s}{s}, \formbytotal{tablecnt}{table}{}{s}{s}, \arabic{bibcount} bibliographic sources and \formbytotal{числоПлакатов}{sheet}{}{s}{s} of graphic material. The number of applications is 2. The graphic material is presented in annex A. The layout of the site, including the connection of components, is presented in annex B.

List of keywords: learning platform, web application, Django, multilingualism, CKEditor, AOS, JavaScript, courses, lessons, tests, achievements, administrator, user, web interface.

The object of development is a web application -- a learning platform designed for computer support of international students' independent work when learning the JavaScript programming language, with a multi-lingual interface and interactive learning tools.

The aim of the final qualification work is to create a convenient and functional platform for independent learning of JavaScript programming language by foreign students, providing access to educational materials, testing and progress tracking through a multi-lingual interface, which contributes to increased involvement and efficiency of learning.

In the process of creating the platform, the main entities were identified by designing data models, Django classes and methods were used to ensure work with entities related to JavaScript learning and correct functioning of the web application, sections for managing courses, lessons, JavaScript tests, progress and achievements analytics were developed, interactive elements were implemented using CKEditor for editing learning content and AOS for interface animation.

The Django framework with integrated libraries for authentication, templating, form processing and multi-language, as well as CKEditor and AOS for interactivity and usability of the learning content were used in the development of the platform.

The developed platform was successfully tested and implemented in the training system.
\selectlanguage{russian}
